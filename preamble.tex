% vim: ts=4 sts=4 sw=4 et
% preamble here.

\documentclass[nofonts, a4paper, oneside, 10pt]{book}

\usepackage{geometry}
\usepackage{amsmath}
\usepackage{fontspec}
\usepackage{upquote}
\usepackage{xeCJK}
\usepackage{hyperref}
\usepackage{setspace}
\usepackage{verbatim}

% 整体页面尺寸
\geometry{ left = 3cm, right = 3cm, bottom = 2cm }

% 设置中文字体
\setCJKmainfont[Scale=0.9, BoldFont=WenQuanYi Zen Hei]{AR PLBaosong2GBK Light}
\setCJKsansfont[Scale=0.9]{WenQuanYi Zen Hei}
\setCJKmonofont[Scale=0.9]{AR PL UMing CN}

% 设置英文字体
\setmainfont{FreeSerif}
\setsansfont{FreeSans}
\setmonofont{FreeMono}

% 中文的 \emph, 'cemph' short for 'chinese emph'
\setCJKfamilyfont{kai}{AR PL UKai CN}
\newcommand\cemph[1]{{\small{\CJKfamily{kai}{#1}}}}

% 为中文设置的双引号
\newenvironment{myquotation}
    {\phantom{ }\CJKsetecglue{}"}
    {"}

% 书中经常出现的词
\newcommand\fahr{{\texttt{fahr}}}

% C 语言关键字
\newcommand\printf{{\texttt{printf}}}
\newcommand\scanf{{\texttt{scanf}}}
\newcommand\cint{{\texttt{int}}}
\newcommand\cfor{{\texttt{for}}}
\newcommand\cwhile{{\texttt{while}}}
\newcommand\cdefine{{\texttt{\#define}}}
\newcommand\cgetchar{{\texttt{getchar}}}
\newcommand\cputchar{{\texttt{putchar}}}
\newcommand\cchar{{\texttt{char}}}
\newcommand\cmain{{\texttt{main}}}
\newcommand\clong{{\texttt{long}}}
\newcommand\cdouble{{\texttt{double}}}
\newcommand\cfloat{{\texttt{float}}}
\newcommand\cif{{\texttt{if}}}
\newcommand\celse{{\texttt{else}}}
\newcommand\cswitch{{\texttt{switch}}}
\newcommand\creturn{{\texttt{return}}}
\newcommand\cvoid{{\texttt{void}}}
\newcommand\cstatic{{\texttt{static}}}
\newcommand\cextern{{\texttt{extern}}}
\newcommand\csigned{{\texttt{signed}}}
\newcommand\cunsigned{{\texttt{unsigned}}}
\newcommand\cconst{{\texttt{const}}}
\newcommand\cshort{{\texttt{short}}}
\newcommand\cenum{{\texttt{enum}}}

% 习题格式
\newcounter{exercnt}
\numberwithin{exercnt}{chapter}
\newcommand\exercise{\stepcounter{exercnt}{\textbf{习题\theexercnt}}}

% 标题信息
\title{The C Programming Lauguage \\ 2nd Edition}
\author{Brian W.Kernighan \and Dennis M.Ritchie}
\date{\today}

% 超链接格式
\hypersetup{
    pdfborder = 0 0 0,              % 超链接无彩色边框
    bookmarksnumbered = true,       % 目录书签带编号
}

% 代码环境
\newenvironment{myverbatim}%
    {\linespread{0.85}\verbatim}%
    {\endverbatim}

% 伪码环境的行距, 与代码环境相同
\newenvironment{mypsudo}%
    {\medskip\begin{spacing}{0.85}}
    {\end{spacing}\medskip}
