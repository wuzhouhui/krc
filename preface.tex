% vim: ts=4 sts=4 sw=4 et
% preface here.

\chapter{前言}
在 \textit{The C Programming Language} 在1978年出版之后, 计算机已经经历了
许多变化. 大型计算机变得得更大, 而个人计算机也拥有了十年前大型机的运算能力.
在这段时间内, C 也有些许变化, 虽然不多, 但是与最初作为UNIX操作系统语言相比
已经, C语言已经传播得非常广泛了.

鉴于C语言变得越来越流行, 并且在这些年间该语言也发生了一些变化, 以及某些组织
开发的编译器, 越来越有必要对语言作出更加精确并且符合现代观点的定义, 与第一
版相比. 1983年, 美国国家标准组织建立了一个委员会,
指在建立一个\begin{myquotation}明确的, 独立%
于机器的C语言\end{myquotation}, 同时保留它的精髓. 结果就是产生的C语言的 
ANSI标准.

标准明确了第一版中提到但没有详细描述的解释, 例如结构与与枚举体赋值. 它提出
一种新的函数定义形式, 该形式允许在使用时交叉检验函数的定义. 它指定了一个标
准函数库, 字符串操作和一个类似的工作. 它使得原来比较模糊的特点更加地明确,
同时显式地说明要保留语言的机器无关性.

\textit{The C Programming Language} 第2版用ANSI标准来描述C语言, 对于该语言
已经演变的部分, 我们会用一种新格式来写出来. 对于大部分, 这并没有什么本质的
区别; 最主要的区别就是函数定义与声明的格式. 现代的编译已经支持标准中的大
多数特性.

我们尽量保持与第1版一样的简洁性. C 不是一种很复杂的语言, 用一本很厚的书籍来
描述它并不合适. 对于C语言的重要特性(例如指针, 它是C语言的精髓)我们会着重
阐述. 我们已经增加并修改了许多示例. 例如, 如何对待复杂的声明是由程序添加的,
该程序会把声明转换到字中, 反之亦然. 与以前一样, 所有的例如都被测试过.

附录A中的参考手册并不是标准, 但是我们想要在更小的空间中传送标准的重要之处.
这于程序员来说这很好理解, 但对于编译器开发者来说这此手册中的内容并不等同于
定义, 定义的责任属于标准. 附录B总结了标准库函数. 同样, 该参考只对程序员有意
义, 而不针对于函数实现人员. 附件C简单地概括了一下自从第一版以来, C语言所发
生的变化.

正如我们在本书第一版前言中所说的那样, "C wears well as on's experience with
it grows". 在十多年的实践中, 我们对这名话感受颇深. 我们希望这本书可以帮助你
学好并用好C语言.

我们非常感谢那些帮助我们写这本书的朋友. Jon Bently, Doug Gwyn, Doug McIlroy,
Peter Nelson 和 Rob Pike 对本书的草稿提出了许多宝贵的意见. 我们也很感谢
Al Aho, Dennis Allison, Joe Campbell, G.R. Emlin, Karen Fortgang, Allen
Holub, Andrew Hume, Dave Kristol, John Linderman, Dave Prosser, Gene 
Spafford 和 Chris van Wyk, 以上这些人非常认真地读了这本书. 我们也从以下这
些人收到了许多有助的建议, 他们是 Bill Cheswick, Mark Kernighan, Andy 
Koenig, Robin Lake, Tom London, Jim Reeds, Clovis Tondo 和 Peter Weinberger.
Dave Prosser 回到了有关ANSI标准的许多细节问题.  我们使用了 Bjarne Strostrup
的 C++翻译器测试了我们的程序, 并且 Dave Kristol 向我们提供了一个 ANSI C 
编译器来做最终的测试. Rich Drechsler 在打字方面帮助很大.

感谢所有的人.
 
Brian W.Kernighan

Dennis M. Ritchie

\chapter{第一版前言}
C语言是一种通用编程语言, 支持表达式, 现代控制流, 数据结构, 以及一系列的操作
符. C语言并不是一种\begin{myquotation}非常高级的编程语言\end{myquotation}, 
它并不\begin{myquotation}大\end{myquotation}, 以不是专门针对某一应用
领域. 由于C语言缺少许多限制以及通用性, 使得它与所谓的更厉害的语言相比, 在
完成许多任务上更加的方便与高效.

C语言最开始是为了在PDP-11上实现UNIX操作系统, 主要是由 Dennis Ritchie 完成
的. 操作系统, C 编译器, 所有重要的 UNIX 应用程序(包括所有的用来写这本书的
软件)都是由 C 开发的. 除此之外, C 还在其他机器上开发了编译器, 包括 IBM
System/370, Honeywell 6000 和 Interdate 8/32. C 语言并不特定了某些硬件或
系统, 然后, 开发那些将要在不同的, 但是支持 C 语言的机器上运行的程序是很容易
的.

这本书指在帮助读者使用 C 语言开发程序. 该书包含一个教程, 能让新手尽快地起步,
根据 C 语言的几个主要特性来划分章节, 另外还有一个参考手册. 读者主要地工作
就是阅读, 写代码和回顾, 而不是仅仅拘泥于语法上. 在大多数情况下, 示例程序都
完整真实的程序, 而不是一个片断. 所有的例子都被测试过. 除了展示如何高效地使
用语言外, 我们也会尽可能地描述一些有用的算法和设计开发风格或原则.

这本书并不是一个介绍编程的手册; 我们假设读者已经拥有一些编程的基本知识, 例
如变量, 赋值语句, 循环和函数. 无论如何, 一个初学者应该有能力阅读下去, 向其
他人寻求帮助也是一个不错的方法.

在我们的实践中, C 已经展现出了它的优雅, 丰富和多功能. 它很容易学习, 实践越
丰富, 它就表现地越好. 我们希望这本书可以帮助你很好地使用它.

许多意见与建议已经添加进了这本书, 而且我们也很乐于这么做. 尤其是 Mike 
Bianchi, Jim Blue, Stu Feldman, Doug McIlroy Bill Roome, Bob Rosin 和 Larry
Rosler, 他们都非常认真地阅读了资料. 我们同时也感谢 Al Aho, Steve Bourne, 
Dan Dvorak, Chuck Haley, Debbie Haley, marion harris, Rick Holt, Steve 
Johnson, John Mashey, Bob Mitze, Ralph Muha, Peter Nelson, Elliot Pinson, 
Bill plauger, Jerry Spivack, Ken Thompson 和 Perter Weinberger, 他们在不同
阶段都提出了宝贵的意见. 同时也感谢 Mile Lesk 和 Joe Ossanna 的帮助与打字.

Brian W. Kernighan

Dennis M. Ritchie
